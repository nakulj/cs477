\newcommand{\risk}[6]{
	\rule{\textwidth}{.1pt}
	\textbf{#1} \textit{Current Rank:} #2 \textit{Previous Rank:} #3 \\
	\paragraph{Risk Description} #4
	\paragraph{Risk Exposure} #5
	\paragraph{Mitigation Action Items} #6\\
}

\newcommand{\nrisk}[5]{
	\rule{\textwidth}{.1pt}
	\textbf{#1} \textit{Current Rank:} #2 \textit{Previous Rank:} #3 \\
	\paragraph{Risk Description} #4
	\paragraph{Mitigation Action Items} #5\\
}

\subsection{Major Risks}

\risk{Network Risk}{1}{N/A}{
Since our app will require a live connection to TAP/LA Metro infrastructure to access ticket prices, and send, receive, and verify data, a weak internet signal can cripple the app's functionality. This can result in lost revenue from users who cannot use the app because of weak network connectivity. This is largely an “external” risk beyond our control.
}{
As presented in the group’s ARB session, below is a risk exposure calculation to estimate revenue loss per month from network connectivity risks.
% Please remember to add \use{multirow} to your document preamble in order to suppor multirow cells
% Booktabs require to add \usepackage{booktabs} to your document preamble
\begin{table}[h]
\begin{tabularx}{\textwidth}{@{}lXrrr@{}}
\toprule
\multirow{2}{*}{} & \multirow{2}{*}{Item}                         & \multicolumn{3}{c}{Exposure}                                                                   \\ \cmidrule(l){3-5} 
                  &                                               & \multicolumn{1}{l}{High +20\%} & \multicolumn{1}{l}{Normative} & \multicolumn{1}{l}{Low -20\%} \\ \midrule
A                 & Avg. number of passengers/day                 & 162,333                        & 135,278                       & 108,222                       \\
B                 & Avg. ticket price                             & \multicolumn{3}{c}{\$2.00}                                                                     \\
C                 & Estimated \% of passengers affected           & 2.4\%                          & 2.0\%                         & 1.6\%                         \\
D                 & Estimated number of passengers affected (A$\times$C) & 3,896                          & 2,706                         & 1,732                         \\
E                 & Estimated revenue loss/day (B$\times$D)              & \$7,792.0                      & \$5411.1                      & 3,463.1                       \\
F                 & Days/month                                    & \multicolumn{3}{c}{30}                                                                         \\
\textbf{G}        & \textbf{Total expected loss/month (E$\times$F)}      & \textbf{\$233,760}             & \textbf{\$162,333}            & \textbf{\$103,893}            \\
H                 & Probability of network issues                 & 96\%                           & 80\%                          & 64\%                          \\
\textbf{}         & \textbf{Total monthly risk exposure (G$\times$H)}    & \textbf{\$224,410}             & \textbf{\$129,867}            & \textbf{\$66,492}            
\\
\bottomrule
\end{tabularx}
\end{table}
}{
The app will inform the user of connectivity errors and show a streamlined version of the app that does not require connectivity. For example, it will show ticket price quotes from the last update, main route availabilities, and transaction history. If possible, the group will work with LA Metro to develop ticket checking systems and protocol on the app without using internet, or via alternative methods such as Bluetooth.
}

\risk{Technology Risk}{2}{3}{
A technical malfunction of our platform might cause boarding delays for passengers and cause inefficiencies in LA Metro operations. For example, if the app fails to update prices or routes, passengers may seek alternative transportation, which translates to lost revenue for the LA Metro.
}{
As presented in the group’s ARB session, below is a risk exposure calculation to estimate revenue loss for an instance of realized technology risk.
% Please remember to add \use{multirow} to your document preamble in order to suppor multirow cells
% Booktabs require to add \usepackage{booktabs} to your document preamble
\begin{table}[h]
\begin{tabularx}{\textwidth}{@{}lXrrr@{}}
\toprule
\multirow{2}{*}{} & \multirow{2}{*}{Item}                         & \multicolumn{3}{c}{Exposure}                                                                               \\ \cmidrule(l){3-5} 
                  &                                               & \multicolumn{1}{l}{High Case + 20\%} & \multicolumn{1}{l}{Normative} & \multicolumn{1}{l}{Low Case - 20\%} \\
            \midrule
A                 & Avg. Number of Passengers/ Day                & 162,333                              & 135,278                       & 108,222                             \\
B                 & Avg. Ticket Price                             & \multicolumn{3}{c}{\$2.0}                                                                                  \\
C                 & Estimated \% of Passengers affected           & 60.0\%                               & 50.0\%                        & 40.0\%                              \\
D                 & Estimated Number of Passengers affected (A*C) & 97,400                               & 67,639                        & 43,289                              \\
E                 & Estimated Revenue Loss/ Day (B*D)             & \$194,800.0                          & \$135,277.8                   & \$86,577.8                          \\
F                 & Estimated number of days to fix problem       & 4.8                                  & 4                             & 3.2                                 \\
\textbf{G}        & \textbf{Total Expected Loss (E*F)}            & \textbf{\$935,040}                   & \textbf{\$541,111}            & \textbf{\$277,049}                  \\
H                 & Probablity of Technology malfuction           & 6.0\%                                & 5.0\%                         & 4.0\%                               \\
\textbf{}         & \textbf{Total Risk Exposure (G*H)}            & \textbf{\$56,102}                    & \textbf{\$27,056}             & \textbf{\$11,082}       \\
\bottomrule           
\end{tabularx}
\end{table}
}{
The group will implement non-normative functionality for certain “big” cases of technology risk. For example, if ticket price update functions are inhibited, the app will charge the user a pre-set fare that is lower than the average rate to compensate for the technology malfunction. Also, there will be a disclaimer advising time-sensitive passengers to have a back-up physical TAP card.
}

\risk{Security Risk}{3}{1}{
An unsecure platform can be exploited (e.g. loopholes to avoid fees, reusing tickets, etc.). An unsecure payment process might compromise user accounts, user information, or LA Metro information. These problems can cut into LA Metro revenue.
}{
As presented in the group’s ARB session, below is a risk exposure calculation to estimate revenue loss for an instance of realized security risk.
% Please remember to add \use{multirow} to your document preamble in order to suppor multirow cells
% Booktabs require to add \usepackage{booktabs} to your document preamble
\begin{table}[h]
\begin{tabularx}{\textwidth}{@{}lXrrr@{}}
\toprule
\multirow{2}{*}{} & \multirow{2}{*}{Item}                         & \multicolumn{3}{c}{Exposure}                                                                               \\ \cmidrule(l){3-5} 
                  &                                               & \multicolumn{1}{l}{High Case + 20\%} & \multicolumn{1}{l}{Normative} & \multicolumn{1}{l}{Low Case - 20\%} \\ \midrule
A                 & Avg. Number of Passengers/ Day                & 162,333                              & 135,278                       & 108,222                             \\
B                 & Avg. Ticket Price                             & \multicolumn{3}{c}{\$2.0}                                                                                  \\
C                 & Estimated \% of Passengers involved           & 24.0\%                               & 20.0\%                        & 16.0\%                              \\
D                 & Estimated Number of Passengers involved (A*C) & 38,960                               & 27,056                        & 17,316                              \\
E                 & Estimated Revenue Loss/ Day (B*D)             & \$77,920.0                           & \$54,111.1                    & \$34,631.1                          \\
F                 & Estimated number of days to fix problem       & 12                                   & 10                            & 8                                   \\
\textbf{G}        & \textbf{Total Expected Loss (E*F)}            & \textbf{\$935,040}                   & \textbf{\$541,111}            & \textbf{\$277,049}                  \\
H                 & Probablity of Security issue                  & 1.2\%                                & 1.0\%                         & 0.8\%                               \\
\textbf{}         & \textbf{Total Risk Exposure (G*H)}            & \textbf{\$11,220}                    & \textbf{\$5,411}              & \textbf{\$2,216}                 
\\ \bottomrule  
\end{tabularx}

\end{table}
}{
Our group’s architect will choose an existing encryption system for our application that will be tested on our app until it meets the client’s requirements and security standards.
}

\risk{Platform Compatibility Risk}{4}{8}{
The group will be using PhoneGap to implement the mobile application with standard web technologies such as HTML/CSS/JavaScript which makes the application subject to device-specific mobile browser display discrepancies. Performance inconsistencies may require additional costs to fix.
}{
As presented in the group’s ARB session, below is a risk exposure calculation to estimate the cost to LA Metro/TAP to fix an escalated instance of realized platform risk.
% Please remember to add \use{multirow} to your document preamble in order to suppor multirow cells
% Booktabs require to add \usepackage{booktabs} to your document preamble
\begin{table}[h]
\begin{tabularx}{\textwidth}{@{}lXrrr@{}}
\toprule
\multirow{2}{*}{} & \multirow{2}{*}{Item}                           & \multicolumn{3}{c}{Exposure}                                                                               \\ \cmidrule(l){3-5} 
                  &                                                 & \multicolumn{1}{l}{High Case + 20\%} & \multicolumn{1}{l}{Normative} & \multicolumn{1}{l}{Low Case - 20\%} \\ \midrule
A                 & Estmated Hourly Rate for IT contractor          & \$42.0                               & \$35.0                        & \$28.0                              \\
B                 & Estimated Team Size                             & 4                                    & 3                             & 2                                   \\
C                 & Estimated Fees for 8 hour day (8*A*B)           & \$1,209.6                            & \$840.0                       & \$537.6                             \\
D                 & Estimated number of days to tweak compatibility & 10                                   & 8                             & 6                                   \\
\textbf{E}        & \textbf{Total Expected Loss (C*D)}              & \textbf{\$11,612}                    & \textbf{\$6,720}              & \textbf{\$3,441}                    \\
F                 & Probablity of compatibility issues              & 12.0\%                               & 10.0\%                        & 8.0\%                               \\
\textbf{}         & \textbf{Total Risk Exposure (E*F)}              & \textbf{\$1,393}                     & \textbf{\$672}                & \textbf{\$275}              
\\ \bottomrule       
\end{tabularx}
\end{table}
}{
The app will be tested on a wide variety of devices (different iPhone versions, windows phone and various Android phones) to ensure the application is rendered properly on all screen sizes and OS versions.

The group has researched the option of developing natively for each platform but it is not feasible given time, costs, and expertise constraints. The group has tested simpler PhoneGap apps on various phones, which ran better and more smoothly than expected, resulting in lower Platform compatibility risk exposure (from first ARB). 
}

\risk{Changing Requirements Risk}{5}{6}{
After the platform has been implemented and integrated, the LA Metro/TAP requirements might change, or the technologies being used in relevant processes might change. Such scenarios may require additional costs to modify the app.
}{
As presented in the group’s ARB session, below is a risk exposure calculation to estimate the cost to LA Metro/TAP to fix an instance of realized changing requirements risk.
\section{Requirements}
\subsection{Capability Requirements}

	\subsubsection{Platform}\begin{enumerate}
		\item The application shall be compatible with the iOS and Android mobile platforms.
		\item The application shall support a web interface for other mobile devices.
	\end{enumerate}
	
	\subsubsection{User Accounts}\begin{enumerate}
		\item If it is a user’s first-time opening the application, the application shall prompt the user to enter an email address, password, and credit card information.
		\begin{enumerate}
			\item If the account is created successfully, the application shall send a confirmation to the user’s stored email address.
			\item If the account creation is not successful, the application shall display an error message that prompts the user to reenter their information and will not be logged in.
		\end{enumerate}
		\item If a returning user opens the application, it shall prompt the user to enter their log-in information.
			\begin{enumerate}
				\item If the log-in attempt fails, the application will display an error message that prompts the user to reenter their log-in information.
				\item If the log-in attempt succeeds, the user will be shown a transit management screen.
			\end{enumerate}
	\end{enumerate}
	
	\subsubsection{Usage}\begin{enumerate}
		\item If the application loses connectivity, the application shall display an error message. 
		\item The application shall retrieve the user’s location at intervals of $120\pm10$ seconds.
		\item If the user is within 150 feet of a train-stop’s geo-location coordinates, the train-stop’s name, ticket price, and incoming train information shall be displayed on the user interface.
		\item If train stop information is being displayed on the application’s user interface, the application shall update the train stop’s incoming train information at intervals of $30\pm 5$ seconds.
	\end{enumerate}
	
	\subsubsection{Payments}\begin{enumerate}
		\item If the user selects to purchase a ticket, their stored credit card will be charged for the price of the train stop’s ticket. If the card does not process, an error message shall by display and the order shall not be accepted. If the transaction succeeds, the application shall create a virtual ticket.
		\item If the user selects to use their ticket, the application shall signal the gate up to a maximum of 3 attempts at $20\pm 5$ second intervals to open. If the application is unsuccessful on the 3rd attempt, an error message shall be displayed.
		\item The application shall store unused virtual tickets for at least 1 year.
		\item The application shall store used virtual tickets for at least 30 days.
		\item The application shall allow the user to change their email address.
		\item The application shall allow the user to change their password.
		\item The application shall allow the user to change their credit card information.
	\end{enumerate}		
\newpage	
\subsection{Level of Service Requirements}

\begin{table}[h]
    \begin{tabularx}{\textwidth}{Xll}
    \hline
    LOS Requirements                                                                                          & Desired Level & Accepted Level \\ \hline
    LOS-1: Concurrent Users                                                                                   & $150000$         & $75000$           \\
    LOS-2: Start-up and user location time                                                                    & 7             & 15             \\
    LOS-3: Ticket Purchase Transaction Time                                                                   & 5             & 20             \\
    LOS-4: Update Account Information Time                                                                    & 10            & 30             \\
    LOS-5: Tickets stored per user                                                                            & 1500          & 500            \\
    LOS-6: \% first-time users able to purchase ticket without outside help                                    & 99            & 95             \\
    LOS-7: \% first-time users able to use ticket without outside help                                         & 99            & 95             \\
    LOS-8: \% users that ride metro at least once per week that would rate ease of use at 3 out of 5 or higher & 80            & 75             \\
    LOS-9: Average Time for User to Create an Account                                                         & 45            & 60             \\
    LOS-10: Failed Ticket Purchases per 1000                                                                  & 1             & 5              \\
    LOS-11: Failed Ticket Uses per 1000                                                                       & 1             & 5              \\
    LOS-12: Hours per day that app shall purchase tickets                                                     & 22            & 20             \\
    LOS-13: Hours per day that app shall allow use of tickets                                                 & 22            & 20             \\
    LOS-14: \# iOS generations app shall support                                                               & 3             & 2              \\
    LOS-15: \# Android generations app shall support                                                           & 3             & 2              \\
    LOS-16: \# versions of app that Metro system shall support                                                 & 3             & 2              \\
    \hline
    \end{tabularx}
\end{table}
}{
If the team is under a contract/agreement, the team will develop updates/patches that will adapt to new requirements. Otherwise, we will give the LA Metro a copy of our comprehensive documentation for the software so that necessary changes can be handled in-house or outsourced to another contractor.

The group has conducted a survey to evaluate currently prototyped features. Good feedback was received from a representative sample size of LA Metro users, which should lower requirements risk exposure (relative to first ARB).
}

\clearpage
\subsection{Minor Risks}

\nrisk{Partnership Risks}{6}{6}{
LA Metro might not be willing to implement our app with their system, or allow us to test compatibility with their system.
}{
Our project manager is currently in talks with the LA Metro authorities, but we could also potentially do a proof of concept using similar hardware/systems. TAP could also be considered a potential partner, as LA Metro outsources projects such as these to TAP.

	Shifted targeting focus to TAP instead of LA Metro. E-mailed TAP to discuss plans moving forward.
}

\nrisk{Estimation Risks}{7}{5}{
The estimated mount of work might not be completed in two semesters, or another developer might implement the system first.
}{
We will develop a running list of unexpected risks as they come up and develop strategies to approach them. For example, we would explicitly ask LA Metro for their timeline and ask for an exclusive partnership.

Utilized Asana as project planning tool to plan advanced deadlines, deliverables, and responsibilities.
}

\nrisk{Expertise Risks}{8}{7}{
Team members might lack necessary expertise in certain development requirements.
}{
Identify team member skills and project requirements so we can individually prepare ourselves for technologies that we will be implementing next semester

Researched and found software tools such as PhoneGap that can help us create a functional app more easily.
}

\nrisk{Organizational Risks}{9}{9}{
Statically defined team roles might not fit specific project requirements.
}{
Dynamically reassign team roles as the project and its requirements mature.  Also the team shall be aware that responsibilities will be very flexible and we may need to step outside the scope of our assigned position.

Assigned responsibilities dynamically via Asana and scheduled more frequent group meetings.
}

