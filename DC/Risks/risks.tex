\newcommand{\risk}[5]{
	\rule{\textwidth}{.1pt}
	\textbf{ Risk\# #1}(Last week: #2)\\
	\textbf{Weeks Active} #3
	\paragraph{Risk Description} #4
	\paragraph{Mitigation Action Items} #5\\
}



\subsection{Technical Risks}

\risk{1}{1}{7}{
An unsecure platform might be exploited or easily manipulated (e.g. people finding loopholes to avoid fees), which can result in the LA Metro losing profits.
}{
Our group’s architect will develop a cryptosystem for our application that will be tested until it meets the client’s requirements and security standards.
}
\risk{3}{3}{7}{
A technical malfunction of our platform might cause boarding delays for passengers and cause inefficiencies in LA Metro operations.
}{
A back-up system will be designed in a later stage of the project. Also, we will advise passengers who are particularly sensitive to delays to have a back-up physical tap card.
}
\risk{4}{4}{7}{
Not enough end users (passengers) might not have phones on iOS and Android systems that are compatible with our app.
}{
A feasibility study will be conducted to verify iOS/Android penetration rates within the LA Metro passenger base, and if needed, we can tailor our platform to target only a certain segment of the customer base that has access to the required technology.
}
\risk{8}{8}{7}{
We will be using PhoneGap to implement the mobile application with standard web technologies like HTML/CSS/Javascript which makes the application subject to device-specific mobile browser display discrepencies.
}{
Test on a wide variety of devices within iOS/Android space (different iPhone versions and various Android phones) to ensure the application is rendered properly on all screen sizes and OS versions.
}

\subsection{Requirement Risks}
\risk{2}{2}{7}{
TAP might not be willing to implement our system or work with our group to test our platform on their systems/hardware.
}{
Our project manager is currently in talks with TAP, but we could also do a proof of concept using similar hardware/systems or mock interfaces.
}
\risk{6}{6}{7}{
After the platform has been implemented and integrated, the TAP requirements might change, or the technologies being used in relevant processes might change.
}{
If we are under a contract/agreement, the team will develop updates/patches that will adapt to new requirements. Otherwise, we will provide a copy of our comprehensive documentation for the software so that necessary changes can be handled in-house.
}
\risk{5}{5}{7}{
To get the application fully working and integrated into TAP/LA Metro infrastructure, there will be a large set of evolving requirements. Two semesters might not be enough time to complete the project, especially if there are unexpected risks (e.g. another developer completes the same project before we do, bureaucratic risks, etc.)
}{
We will develop a running list of unexpected risks as they come up and develop strategies to approach them. For example, we would explicitly ask TAP for their timeline and ask for an exclusive partnership.
}

\subsection{Human Resources Risks}

\risk{7}{7}{7}{
Due to rapid formation of teams without proper analysis of required skill sets needed for this project we may lack the technical skills to complete this project in a professionally acceptable manner.
}{
Identify team member skills and project requirements so we can individually prepare ourselves for technologies that we will be implementing next semester.
}
\risk{9}{9}{7}{
The statically defined list of team roles may not fit our specific project requirements.
}{
Dynamically reassign team roles as the project and its requirements mature.  Also the team shall be aware that responsibilities will be very flexible and we may need to step outside the scope of our assigned position
}
\risk{10}{10}{7}{
The amount of cooperation needed between our team and TAP may be more than expected which could make this project unfeasible for them.}
{
Come up with a plan focused around the idea that TAP should have to do as little as possible.  Extensive testing should be done internally before our software can be ready for beta testing with TAP.  The goal should be to get it right the first time.
}
\rule{\textwidth}{.1pt}
%\begin{landscape}
%\begin{table}[h]
\begin{center}
\begin{tabularx}{\textwidth}{cccXXc}

1 & N/A & Security & An unsecure platform might be exploited or easily manipulated (e.g. people finding loopholes to avoid fees), which can result in the LA Metro losing profits. & Our group�s architect will develop a cryptosystem for our application that will be tested until it meets the client�s requirements and security standards. & 1 \\ 
2 & N/A & Partnership & The LA Metro might not be willing to implement our system or work with our group to test our platform on their systems/hardware. & Our project manager is currently in talks with the LA Metro authorities, but we could also potentially do a proof of concept using similar hardware/systems. & 1 \\ 
3 & N/A & Technology & A technical malfunction of our platform might cause boarding delays for passengers and cause inefficiencies in LA Metro operations. & A back-up system will be designed in a later stage of the project. Also, we will advise passengers who are particularly sensitive to delays to have a back-up physical tap card. & 1 \\ 
4 & N/A & Technology & End users (passengers) might not have smartphones that have the technical capabilities to support our platform. & A feasibility study will be conducted to verify compatible phone penetration rates, and if needed, we can tailor our platform to target only a certain segment of the customer base that has access to the required technology. & 1 \\ 
5 & N/A & Estimation & To get the application fully working and integrated into the LA Metro system, there will be a large set of evolving requirements. Two semesters might not be enough time to complete the project, especially if there are unexpected risks (e.g. another developer completes the same project before we do, bureaucratic risks, etc.) & We will develop a running list of unexpected risks as they come up and develop strategies to approach them. For example, we would explicitly ask LA Metro for their timeline and ask for an exclusive partnership. & 1 \\ 
6 & N/A & Requirements & After the platform has been implemented and integrated, the LA Metro requirements might change, or the technologies being used in relevant processes might change. & If we are under a contract/agreement, the team will develop updates/patches that will adapt to new requirements. Otherwise, we will give the LA Metro a copy of our comprehensive documentation for the software so that necessary changes can be handled in-house. & 1 \\ 
7 & N/A & People & Due to rapid formation of teams without proper analysis of required skill sets needed for this project we may lack the technical skills to complete this project in a professionally acceptable manner.
 & Identify team member skills and project requirements so we can individually prepare ourselves for technologies that we will be implementing next semester. & 1 \\ 
8 & N/A & Tools & We will be using PhoneGap to implement the mobile application with standard web technologies like HTML/CSS/Javascript which makes the application subject to device-specific mobile browser display discrepencies. & Test on a wide variety of devices (different iPhone versions and various Android phones) to ensure the application is rendered properly on all screen sizes and OS versions. & 1 \\ 
9 & N/A & Organizational & The statically defined list of team roles may not fit our specific project requirements. & Dynamically reassign team roles as the project and its requirements mature.  Also the team shall be aware that responsibilities will be very flexible and we may need to step outside the scope of our assigned position & 1 \\ 
10 & N/A & Estimation & The amount of cooperation needed between our team and LA Metro may be more than expected which could make this project unfeasible for them. & Come up with a plan focused around the idea that LA Metro should have to do as little as possible.  Extensive testing should be done internally before our software can be ready for LA Metro.  The goal should be to get it right the first time. & 1 \\ 
\end{tabularx}
\end{center}
\end{table}

%\end{landscape}
