\documentclass{article}
\usepackage{hyperref}
\title{CS 477a Project Proposal: \\Mobile Transit Passes}
\date{}
\author{
	Andrew Borba,
	Lizz Brooks,\\
	Jorge Go,
	Katherine Hu,
	Nakul Joshi,\\
	Ian Malave,
	Rishi Mukhopadyay
}

\begin{document}
\maketitle

\section{Proposed Project Name}
Project RAP (Rapid Access Pass)

\section{Organisation Name}
Mobile Ticketing Technologies

\section{Project Goals}
\begin{enumerate}
	\item To create new and innovative transport technologies
	\item To improve the efficiency of purchasing public transport tickets
	\item To reduce the barriers to public transit systems
\end{enumerate}

\section{Current Systems}
	\subsection{System Names}
		The current available systems include TAP\footnote{\url{http://www.taptogo.net/}} and Clipper\footnote{\url{http://www.clippercard.net/}} cards.
	\subsection{Implementations}
		Current systems are implemented as tickets on plastic 'smart cards', combined with RFID or NFC technologies.
	\subsection{Issues}
		The cards are inconvenient to carry around and not very efficient. Today's technologies are moving towards mobile, which would be more cost-effective for businesses and more attractive to customers.

\section{Proposed Alternative System}
	We propose developing an ecosystem of mobile apps and hardware that would support saving of tickets on users' own mobile devices.
	\subsection{Benefits to clients}
		\begin{enumerate}
			\item Mobile passes would be more cost-effective for the transit operators as they would not have to bear the cost of manufacturing smart cards.
			\item Transit operators could easily verify that passengers have paid for rides.
			\item More convenient to carry around as people have their phone on them most of the time.
			\item 
		\end{enumerate}
\end{document}